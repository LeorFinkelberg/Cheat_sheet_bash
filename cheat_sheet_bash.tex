\documentclass[%
	11pt,
	a4paper,
	utf8,
	%twocolumn
		]{article}	

\usepackage{style_packages/podvoyskiy_article_extended}


\begin{document}
\title{Наиболее полезные конструкции командной оболочки Bash}

\author{}

\date{}
\maketitle

\thispagestyle{fancy}

%\tableofcontents


\section{Конструкции оболочки bash}

Найти в корневом каталоге и всех подкатлогах (\texttt{/}), обычные файлы (\texttt{-type f}), измененные за последний день (\texttt{-mtime -l}), за исключением тех файлов, у которых есть суффикс \texttt{.o} (\texttt{! -name '*.o'})

\begin{lstlisting}[
language = cmd,
numbers = none
]
find / -type f -mtime -l ! -name '*.o'
\end{lstlisting}

Вывод имен файлов и удаление файлов с именами \texttt{core} или \texttt{junk} из рабочего каталога и всех его подкаталогов (круглые скобки обязательно отделяются пробелами)

\begin{lstlisting}[
language = cmd,
numbers = none
]
find . \( -name core -o -name junk \) -print -exec rm {} \;
\end{lstlisting}

Скопировать все \texttt{csv}-файлы из родительской директории (\texttt{..}) в текущую (\texttt{.})

\begin{lstlisting}[
language = cmd,
numbers = none
]
cp -ip ../*.csv
\end{lstlisting}

Скопировать файл из родительской директории в текущую директорию

\begin{lstlisting}[
language = cmd,
numbers = none
]
cp -ip ../Cheat_sheet_Git/cheat_sheet_git.tex .
\end{lstlisting}

Скопировать одну директорию в другую

\begin{lstlisting}[
language = cmd,
numbers = none
]
cp -rip ../Cheat_sheet_Git/style_packages/ .
\end{lstlisting}

Переименовать файл

\begin{lstlisting}[
language = cmd,
numbers = none
]
mv cheat_sheet_git.tex cheat_sheet_bash.tex
\end{lstlisting}

Найти все файлы с расширением \texttt{*.csv} и выбрать из них те, в которых содержится строка \texttt{'state'} (для каждого файла, отвечающего поисковому шаблону, запускается свой процесс)

\begin{lstlisting}[
language = cmd,
numbers = none
]
find . -name '*.csv' -exec grep -niE 'state' {} \;
\end{lstlisting}

Вывести список файлов из текущей директории и всех поддиректорий

\begin{lstlisting}[
language = cmd,
numbers = none
]
ls -l *
\end{lstlisting}

Найти среди файлов с расширением \texttt{*.py} те, в именах которых есть подстрока \texttt{'spark'} (используется конвейер)

\begin{lstlisting}[
language = cmd,
numbers = none
]
ls -l *.py | grep -iE 'spark'
\end{lstlisting}

Найти файлы с расширением \texttt{*.py} и к каждому из них применить команду \texttt{grep}, которая будет искать в файле подстроку \texttt{'argparse'} без учета регистра, с выводом номера строки, на которой она нашла искомую строку по регулярному выражению \texttt{'argparse'} (работает {\color{deepred} медленно}, так как для каждого файла, отвечающего поисковому шаблону запускается свой процесс)

\begin{lstlisting}[
language = cmd,
numbers = none
]
find . -maxdepth 1 -name '*.py' -exec grep -iE 'argparse' {} \;
\end{lstlisting}

Альтернативный вариант с использованием \texttt{xargs} (работает значительно быстрее варианта с \texttt{-exec})

\begin{lstlisting}[
language = cmd,
numbers = none
]
find . -maxdepth 1 -name '*.py' | xargs grep -inE 
\end{lstlisting} 

Найти в файлах с расширением \texttt{*.tex} строку \texttt{'section'} без учета регистра и вывести три строки контекста
\begin{lstlisting}[
language = cmd,
numbers = none
]
find . -name '*.tex' | xargs grep -iE 'section' -3
\end{lstlisting}

Получить информацию о доступном метсе на диске
\begin{lstlisting}[
language = cmd,
numbers = none
]
df -h
\end{lstlisting}


% Источники в "Газовой промышленности" нумеруются по мере упоминания 
\begin{thebibliography}{99}\addcontentsline{toc}{section}{Список литературы}
	\bibitem{ Sobel-2011 }{ Собель М. Linux. Администрирование и системное программирование. 2-е изд. -- СПб.: Питер, 2011. -- 880 с. }
\end{thebibliography}

%\listoffigures\addcontentsline{toc}{section}{Список иллюстраций}

\end{document}
